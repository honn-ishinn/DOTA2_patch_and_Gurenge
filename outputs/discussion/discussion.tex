% Options for packages loaded elsewhere
\PassOptionsToPackage{unicode}{hyperref}
\PassOptionsToPackage{hyphens}{url}
%
\documentclass[
]{article}
\usepackage{lmodern}
\usepackage{amssymb,amsmath}
\usepackage{ifxetex,ifluatex}
\ifnum 0\ifxetex 1\fi\ifluatex 1\fi=0 % if pdftex
  \usepackage[T1]{fontenc}
  \usepackage[utf8]{inputenc}
  \usepackage{textcomp} % provide euro and other symbols
\else % if luatex or xetex
  \usepackage{unicode-math}
  \defaultfontfeatures{Scale=MatchLowercase}
  \defaultfontfeatures[\rmfamily]{Ligatures=TeX,Scale=1}
\fi
% Use upquote if available, for straight quotes in verbatim environments
\IfFileExists{upquote.sty}{\usepackage{upquote}}{}
\IfFileExists{microtype.sty}{% use microtype if available
  \usepackage[]{microtype}
  \UseMicrotypeSet[protrusion]{basicmath} % disable protrusion for tt fonts
}{}
\makeatletter
\@ifundefined{KOMAClassName}{% if non-KOMA class
  \IfFileExists{parskip.sty}{%
    \usepackage{parskip}
  }{% else
    \setlength{\parindent}{0pt}
    \setlength{\parskip}{6pt plus 2pt minus 1pt}}
}{% if KOMA class
  \KOMAoptions{parskip=half}}
\makeatother
\usepackage{xcolor}
\IfFileExists{xurl.sty}{\usepackage{xurl}}{} % add URL line breaks if available
\IfFileExists{bookmark.sty}{\usepackage{bookmark}}{\usepackage{hyperref}}
\hypersetup{
  pdftitle={Discussion},
  pdfauthor={Hong Shi},
  hidelinks,
  pdfcreator={LaTeX via pandoc}}
\urlstyle{same} % disable monospaced font for URLs
\usepackage[margin=1in]{geometry}
\usepackage{color}
\usepackage{fancyvrb}
\newcommand{\VerbBar}{|}
\newcommand{\VERB}{\Verb[commandchars=\\\{\}]}
\DefineVerbatimEnvironment{Highlighting}{Verbatim}{commandchars=\\\{\}}
% Add ',fontsize=\small' for more characters per line
\usepackage{framed}
\definecolor{shadecolor}{RGB}{248,248,248}
\newenvironment{Shaded}{\begin{snugshade}}{\end{snugshade}}
\newcommand{\AlertTok}[1]{\textcolor[rgb]{0.94,0.16,0.16}{#1}}
\newcommand{\AnnotationTok}[1]{\textcolor[rgb]{0.56,0.35,0.01}{\textbf{\textit{#1}}}}
\newcommand{\AttributeTok}[1]{\textcolor[rgb]{0.77,0.63,0.00}{#1}}
\newcommand{\BaseNTok}[1]{\textcolor[rgb]{0.00,0.00,0.81}{#1}}
\newcommand{\BuiltInTok}[1]{#1}
\newcommand{\CharTok}[1]{\textcolor[rgb]{0.31,0.60,0.02}{#1}}
\newcommand{\CommentTok}[1]{\textcolor[rgb]{0.56,0.35,0.01}{\textit{#1}}}
\newcommand{\CommentVarTok}[1]{\textcolor[rgb]{0.56,0.35,0.01}{\textbf{\textit{#1}}}}
\newcommand{\ConstantTok}[1]{\textcolor[rgb]{0.00,0.00,0.00}{#1}}
\newcommand{\ControlFlowTok}[1]{\textcolor[rgb]{0.13,0.29,0.53}{\textbf{#1}}}
\newcommand{\DataTypeTok}[1]{\textcolor[rgb]{0.13,0.29,0.53}{#1}}
\newcommand{\DecValTok}[1]{\textcolor[rgb]{0.00,0.00,0.81}{#1}}
\newcommand{\DocumentationTok}[1]{\textcolor[rgb]{0.56,0.35,0.01}{\textbf{\textit{#1}}}}
\newcommand{\ErrorTok}[1]{\textcolor[rgb]{0.64,0.00,0.00}{\textbf{#1}}}
\newcommand{\ExtensionTok}[1]{#1}
\newcommand{\FloatTok}[1]{\textcolor[rgb]{0.00,0.00,0.81}{#1}}
\newcommand{\FunctionTok}[1]{\textcolor[rgb]{0.00,0.00,0.00}{#1}}
\newcommand{\ImportTok}[1]{#1}
\newcommand{\InformationTok}[1]{\textcolor[rgb]{0.56,0.35,0.01}{\textbf{\textit{#1}}}}
\newcommand{\KeywordTok}[1]{\textcolor[rgb]{0.13,0.29,0.53}{\textbf{#1}}}
\newcommand{\NormalTok}[1]{#1}
\newcommand{\OperatorTok}[1]{\textcolor[rgb]{0.81,0.36,0.00}{\textbf{#1}}}
\newcommand{\OtherTok}[1]{\textcolor[rgb]{0.56,0.35,0.01}{#1}}
\newcommand{\PreprocessorTok}[1]{\textcolor[rgb]{0.56,0.35,0.01}{\textit{#1}}}
\newcommand{\RegionMarkerTok}[1]{#1}
\newcommand{\SpecialCharTok}[1]{\textcolor[rgb]{0.00,0.00,0.00}{#1}}
\newcommand{\SpecialStringTok}[1]{\textcolor[rgb]{0.31,0.60,0.02}{#1}}
\newcommand{\StringTok}[1]{\textcolor[rgb]{0.31,0.60,0.02}{#1}}
\newcommand{\VariableTok}[1]{\textcolor[rgb]{0.00,0.00,0.00}{#1}}
\newcommand{\VerbatimStringTok}[1]{\textcolor[rgb]{0.31,0.60,0.02}{#1}}
\newcommand{\WarningTok}[1]{\textcolor[rgb]{0.56,0.35,0.01}{\textbf{\textit{#1}}}}
\usepackage{graphicx,grffile}
\makeatletter
\def\maxwidth{\ifdim\Gin@nat@width>\linewidth\linewidth\else\Gin@nat@width\fi}
\def\maxheight{\ifdim\Gin@nat@height>\textheight\textheight\else\Gin@nat@height\fi}
\makeatother
% Scale images if necessary, so that they will not overflow the page
% margins by default, and it is still possible to overwrite the defaults
% using explicit options in \includegraphics[width, height, ...]{}
\setkeys{Gin}{width=\maxwidth,height=\maxheight,keepaspectratio}
% Set default figure placement to htbp
\makeatletter
\def\fps@figure{htbp}
\makeatother
\setlength{\emergencystretch}{3em} % prevent overfull lines
\providecommand{\tightlist}{%
  \setlength{\itemsep}{0pt}\setlength{\parskip}{0pt}}
\setcounter{secnumdepth}{-\maxdimen} % remove section numbering

\title{Discussion}
\author{Hong Shi}
\date{2/2/2021}

\begin{document}
\maketitle

In this data gathering practice, I tried two methods collecting the
data: (1) one through twitter API using \texttt{rtweet} package, and (2)
the other through OCR using \texttt{tesseract} package:

\hypertarget{using-rtweet-package}{%
\section{\texorpdfstring{(1) Using \texttt{rtweet}
package}{(1) Using rtweet package}}\label{using-rtweet-package}}

Get DOTA2 patch update tweets( A popular 5v5 MOBA game. OpenAI Five
launched a machine learning project in this game and the OpenAI system
showed the ability to defeat professional teams. Web URL of the project:
\url{https://openai.com/projects/five/}) ~ I went through following
steps to gather DOTA2 patch update data:

\begin{itemize}
\item
  Since I would like to get tweets information from the official DOTA2
  account, I used the \texttt{get\_timeline()} function and apply to
  user ``DOTA2'' to collect all tweets published from the official
  account (raw data).
\item
  When I wanted to save the raw data into csv file, an error showed that
  \texttt{hashtags} is a list column so I could not save the data to
  csv. I googled a function to convety any list columns to character
  type, and successfully save the csv file after applying the function.
\item
  After my inspection of tweet texts, I notice that not all tweets are
  related to gameplay updates. But as a player of DOTA2, I understand
  the pattern of patch updates
  \texttt{"\textbackslash{}\textbackslash{}\textless{}{[}6,7{]}\textbackslash{}\textbackslash{}.{[}0-9{]}\{2\}.?\textbackslash{}\textbackslash{}\textgreater{}"}
  \footnote{Detailed explanation of this pattern in
    ``inputs/scripts/01\_data2\_patch\_tweet''}, so I could use Regex to
  select tweets related to updates.
\end{itemize}

Thoughts and possible followups: Functions in \texttt{rtweet} require
some understandings of features in tweet ( e.g.: like, follow, etc.) I
could use the dataset to analyze the popularity of each patch.

\hypertarget{using-tesseract-package}{%
\section{\texorpdfstring{(2) Using \texttt{tesseract}
package}{(2) Using tesseract package}}\label{using-tesseract-package}}

\begin{Shaded}
\begin{Highlighting}[]
\KeywordTok{library}\NormalTok{(tesseract)}
\KeywordTok{library}\NormalTok{(tidyverse)}
\KeywordTok{library}\NormalTok{(here)}
\KeywordTok{library}\NormalTok{(bookdown)}
\CommentTok{# tesseract_download("jpn") if Japanese is not installed}
\end{Highlighting}
\end{Shaded}

Get the lyrics of a Japanese song「紅蓮華」 (Gurenge) in txt format. ~ I
went through following steps to gather the lyrics in Japanese:

\begin{itemize}
\item
  I chose an image of with the ``Gurenge'' lyrics ( Source from Twitter)
\item
  I noticed that Japanese language was not installed in
  \texttt{tesseract} package, so I installed the Japanese language
  engine and extract the text from it. And it seems that
  \texttt{tesseract} recognizes Japanese characters relatively poor (
  falsely recognize the Kanji ``震'' as ``選'', ``睨'' as ``明'' and
  completely mess up the Hiragana sentence ``どうしたって'' as
  ``2357つうsGgl'' ).
\end{itemize}

\begin{verbatim}
## [1] "UTF-8"
\end{verbatim}

\begin{verbatim}
## mlau<U+4EE4>                   0:54        $@<U+30A4>@( 69%(<U+8A08><U+30EA><U+30BF>
## <U+304F><U+30E1>><U+30E2>                               @<U+3002> <U+4E2D>
## 
## <U+7D05><U+84EE><U+83EF> <U+6B4C><U+8A5E>
## 
## <U+5F37><U+304F><U+306A><U+308C><U+308B><U+7406><U+7531><U+3092><U+77E5><U+3063><U+305F>
## 
## <U+50D5><U+3092><U+9023><U+308C><U+3066> <U+9032><U+3081>
## 
## <U+6CE5><U+3060><U+3089><U+3051><U+306E><U+8D70><U+99AC><U+706F><U+306B><U+9154><U+3046>
## 
## <U+5F37><U+3070><U+308B><U+5FC3> <U+9078><U+3048><U+308B><U+624B><U+306F>
## 
## <U+63B4><U+307F><U+305F><U+3044><U+3082><U+306E><U+304C><U+3042><U+308B>
## 
## <U+305D><U+308C><U+3060><U+3051><U+3055>
## 
## <U+591C><U+306E><U+5302><U+3044><U+306B>(| spend all thirty nights)
## 
## <U+7A7A><U+660E><U+3093><U+3067><U+3082>(Storing into the sky)
## 
## <U+5909><U+308F><U+3063><U+3066><U+3044><U+3051><U+308B><U+306E><U+306F> <U+81EA><U+5206><U+81EA><U+8EAB><U+3060><U+3051>
## 
## <U+305D><U+308C><U+3060><U+3051><U+3055>
## 
## <U+5F37><U+304F><U+306A><U+308C><U+308B><U+7406><U+7531><U+3092><U+77E5><U+3063><U+305F>
## 
## <U+50D5><U+3092><U+9023><U+308C><U+3066> <U+9032><U+3081>
## 
## 2357<U+3064><U+3046>sGgl
## 
## <U+6D88><U+305B><U+306A><U+3044><U+5922><U+3082> <U+6B62><U+307E><U+308C><U+306A><U+3044><U+4ECA><U+3082>
## 
## <U+8AB0><U+304B><U+306E><U+305F><U+3081><U+306B><U+5F37><U+304F><U+306A><U+308C><U+308B><U+306A><U+3089>
## 
## <U+3042><U+308A><U+304C><U+3068><U+3046> <U+60B2><U+3057><U+307F><U+3088>
## 
## <U+4E16><U+754C><U+306B><U+6253><U+3061><U+306E><U+3081><U+3055><U+308C><U+3066>
## 
## <U+89D2><U+3051><U+308B><U+610F><U+5473><U+3092><U+77E5><U+3063><U+305F>
## 
##          <U+3007>         <U+3007>         <U+3081>         <U+30F2>
## 
\end{verbatim}

\begin{itemize}
\tightlist
\item
  (This made me suffer almost a whole night figuring out the solution)
  After getting the text of lyrics, I would like to save it to a txt
  file. However, the output text file is shown below, even though I
  wanted the text file to match what I show in R console:
\end{itemize}

\begin{verbatim}
## mlau<U+4EE4>                   0:54        $@<U+30A4>@( 69%(<U+8A08><U+30EA><U+30BF>
## <U+304F><U+30E1>><U+30E2>                               @<U+3002> <U+4E2D>
## 
## <U+7D05><U+84EE><U+83EF> <U+6B4C><U+8A5E>
## 
## <U+5F37><U+304F><U+306A><U+308C><U+308B><U+7406><U+7531><U+3092><U+77E5><U+3063><U+305F>
## 
## <U+50D5><U+3092><U+9023><U+308C><U+3066> <U+9032><U+3081>
## 
## <U+6CE5><U+3060><U+3089><U+3051><U+306E><U+8D70><U+99AC><U+706F><U+306B><U+9154><U+3046>
## 
## <U+5F37><U+3070><U+308B><U+5FC3> <U+9078><U+3048><U+308B><U+624B><U+306F>
## 
## <U+63B4><U+307F><U+305F><U+3044><U+3082><U+306E><U+304C><U+3042><U+308B>
## 
## <U+305D><U+308C><U+3060><U+3051><U+3055>
## 
## <U+591C><U+306E><U+5302><U+3044><U+306B>(| spend all thirty nights)
## 
## <U+7A7A><U+660E><U+3093><U+3067><U+3082>(Storing into the sky)
## 
## <U+5909><U+308F><U+3063><U+3066><U+3044><U+3051><U+308B><U+306E><U+306F> <U+81EA><U+5206><U+81EA><U+8EAB><U+3060><U+3051>
## 
## <U+305D><U+308C><U+3060><U+3051><U+3055>
## 
## <U+5F37><U+304F><U+306A><U+308C><U+308B><U+7406><U+7531><U+3092><U+77E5><U+3063><U+305F>
## 
## <U+50D5><U+3092><U+9023><U+308C><U+3066> <U+9032><U+3081>
## 
## 2357<U+3064><U+3046>sGgl
## 
## <U+6D88><U+305B><U+306A><U+3044><U+5922><U+3082> <U+6B62><U+307E><U+308C><U+306A><U+3044><U+4ECA><U+3082>
## 
## <U+8AB0><U+304B><U+306E><U+305F><U+3081><U+306B><U+5F37><U+304F><U+306A><U+308C><U+308B><U+306A><U+3089>
## 
## <U+3042><U+308A><U+304C><U+3068><U+3046> <U+60B2><U+3057><U+307F><U+3088>
## 
## <U+4E16><U+754C><U+306B><U+6253><U+3061><U+306E><U+3081><U+3055><U+308C><U+3066>
## 
## <U+89D2><U+3051><U+308B><U+610F><U+5473><U+3092><U+77E5><U+3063><U+305F>
## 
##          <U+3007>         <U+3007>         <U+3081>         <U+30F2>
\end{verbatim}

I understood that the ``\textless U+XXXX\textgreater{}'' is the Unicode
representation of the Japanese character and the text I extracted using
\texttt{tesseract} package is a character variable with Unicodes. I
spent a great amount of time googling and figured out that the text of
the character variable it self is already encoded in ``UTF-8''. So if I
save the encoded ``UTF-8'' text, no matter which encoding method I chose
to open the file, the text file will always display the Unicode text as
shown above. So I need to reencode the text \texttt{tesseract} generated
first, and then save the txt file to make sure I could display the exact
Japanese characters in .txt file.

However, the output html document ( I ) I would like to get some follow
up analysis about DOTA2 patch updates hopefully in reading week :)

\end{document}
